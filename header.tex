%Schriftgrösse, Layout, Papierformat, Art des Dokumentes
\documentclass[11pt,twoside,a4paper,fleqn]{article}
%Einstellungen der Seitenränder
\usepackage[left=1cm,right=1cm,top=1cm,bottom=1cm,includeheadfoot]{geometry}
% Sprache, Zeichensatz, packages
\usepackage[utf8]{inputenc}
\usepackage[ngerman]{babel,varioref}
\usepackage{amssymb,amsmath,fancybox,graphicx,lastpage,wrapfig,fancyhdr,hyperref,verbatim}


%Farben für tabelle
\usepackage{color, colortbl}
\definecolor{Gray}{gray}{0.9}
\definecolor{LightCyan}{rgb}{0.88,1,1}
%\cellcolor{Farbe} für Farben einer Zelle
%\rowcolor{Farbe}  für Farben einer Zeile

%Schriftarten
%http://www.tug.dk/FontCatalogue/arial/
\renewcommand*\familydefault{\sfdefault} %% Only if the base font of the document is to be sans serif
\usepackage[T1]{fontenc}

%tabularx package
%http://mirror.switch.ch/ftp/mirror/tex/macros/latex/required/tools/tabularx.pdf
\usepackage{tabularx}
%tabularx: Damit inhalt der Tabelle zentriert ist
%\renewcommand{\tabularxcolumn}[1]{>{\normalsize\centering\arraybackslash}m{#1}}
%aufhebender Befehl
%\def\tabularxcolumn#1{p{#1}}


%pdf info
\hypersetup{pdfauthor={\authorinfo},pdftitle={\titleinfo},colorlinks=false}
\author{\authorinfo}
\title{\titleinfo}

%Kopf- und Fusszeile
\pagestyle{fancy}
\fancyhf{}
%Linien oben und unten
\renewcommand{\headrulewidth}{0.5pt} 
\renewcommand{\footrulewidth}{0.5pt}

\fancyhead[L]{\titleinfo{ }\tiny{(\versioninfo)}}
%Kopfzeile rechts bzw. aussen
\fancyhead[R]{Seite \thepage { }von \pageref{LastPage}}
%Fusszeile links bzw. innen
\fancyfoot[L]{\footnotesize{\authorinfo}}
%Fusszeile rechts bzw. ausen
\fancyfoot[R]{\footnotesize{\today}}
