\section{Text 1}
cc: chef-chef, L: lehrling\\
\\
cc: Hallo L, ich habe etwas mit dir zu besprechen. \\
l : Guten Tag Herr cc, wie kann ich Ihnen helfen?\\
cc: Sie haben gestern ihre Arbeit abgeliefert, wie zufrieden sind sie damit?\\
l : Ich hatte schwierigkeiten, aber bin der Meinung, ich habe es gut hinbekommen.\\
cc: Ist das ihr Ernst?\\
l : ääähh\\
cc: sie können ja nicht ernsthaft damit zufrieden sein.\\
l : was meinen sie damit? \\
cc: Als ich diese Arbeit sah, war ich schockiert. Die essentielle Grundlagen waren überhaupt nicht eingehalten. \\
l : Der Schaltplan wurde doch von ihnen korrigiert, oder?\\
cc: natürlich, aber der war auch korrekt und das ist nicht das Problem hier!\\
l : An was liegt es denn?!..\\
cc: Das Layout war ja unter aller Sau. Haben sie denn noch nichts gelernt bei uns?\\
l : \emph{("geschocktes schweigen")}\\
cc: sehen sie sich mal diese Stelle an, sehen sie denn hier den Fehler nicht?\\
l : Ehrlich gesagt, nein. Das wurde mir in der kurzen Zeit nicht erklärt.\\
cc: Ja, aber wie sie zum Beispiel die Winkel der Leiterbahnenecken gemacht haben. Da wird es ja den elektronen schlecht!\\
l : \emph{(verwirrt)}soll ich den Kurven machen?\\
cc: sicher nicht!\\
l : dann sagen sie mir doch wie denn?\\
cc: In welchem Lehrjahr sind sie denn eigentlich?\\
l : In meinem ersten.\\
cc: oh...