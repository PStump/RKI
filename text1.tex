\section{Ausgangsdialog}
\herrcc: Chef-Chef; \herrl: Lehrling\\

\begin{center}
  \begin{tabular}{r p{13cm}}
  \speakcc  Hallo \herrl, ich habe etwas mit dir zu besprechen. \\
  \speakl   Guten Tag Herr Caesar, wie kann ich Ihnen helfen?\\

  \speakcc  Sie haben gestern ihre Arbeit abgeliefert, wie zufrieden sind sie damit?\\
  \speakl   Ich hatte schwierigkeiten, aber bin der Meinung,
              ich habe es gut hinbekommen.\\

  \speakcc  Ist das ihr Ernst?\\
  \speakl   ääähh\\

  \speakcc  sie können ja nicht ernsthaft damit zufrieden sein.\\
  \speakl   was meinen sie damit? \\

  \speakcc  Als ich diese Arbeit sah, war ich schockiert.
              Die essentielle Grundlagen waren überhaupt nicht eingehalten. \\
  \speakl   Der Schaltplan wurde doch von ihnen korrigiert, oder?\\

  \speakcc  natürlich, aber der war auch korrekt und das ist nicht das Problem hier!\\
  \speakl   An was liegt es denn?!..\\

  \speakcc  Das Layout war ja unter aller Sau.
              Haben sie denn noch nichts gelernt bei uns?\\
  \speakl   \emph{("geschocktes schweigen")}\\
  \speakcc  sehen sie sich mal diese Stelle an, sehen sie denn hier den Fehler nicht?\\

  \speakl   Ehrlich gesagt, nein. Das wurde mir in der kurzen Zeit nicht erklärt.\\
  \speakcc  Ja, aber wie sie zum Beispiel die Winkel der Leiterbahnenecken gemacht haben.
              Da wird es ja den elektronen schlecht!\\

  \speakl   \emph{(verwirrt)} soll ich den Kurven machen?\\
  \speakcc  sicher nicht!\\

  \speakl   dann sagen sie mir doch wie denn?\\
  \speakcc  In welchem Lehrjahr sind sie denn eigentlich?\\

  \speakl   In meinem ersten.\\
  \speakcc  Wieso hast Du das nicht von Anfang an gesagt! \emph{(gehässig)}
  \end{tabular}
\end{center}