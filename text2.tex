\section{Dialogverbesserung v0.1}
\herrcc: Chef-Chef; \herrl: Lehrling\\

\subsection{Für Präsentation?}
Für die Verbesserung des Dialoges gebe es zwei kurze Möglichkeiten
\begin{enumerate}
  \item Man frage, bevor man irgend etwas (unbekanntes) hin schustert.
  \item Man versuche so schnell wie möglich klar zu stellen,
          dass man diese Arbeit noch nie gemacht hat. (Was an sich wieder
          zum 1.\,ten Punkt führt.)
\end{enumerate}

Diese beiden Wege sind zwar gut und vor allem der 1.\,te auch schmerzlos.
$\rightarrow$ sehr Effektiv / schnell Ziehlführend (Weissbach Kapitel 1/2?)

Für diesen 2.\,ten Teil dieser Präsentation haben wir uns jedoch für einen anderen
Weg entschieden.  Mit Hilfe von Kapitel 18 aus dem Weissbach, haben wir versucht
den Dialog besser zu gestallten.

Dazu haben wir den Lehrling \herrl mit dem Weissbachwissen ausgestattet.  Man kann
sich natürlich berechtigterweise fragen, ob ein Lehrling des nötige \glqq Selbstvertrauen\grqq 
gegenüber dem Chef-Chef Herrn \herrcc aufbringen vermag.  Wir denken jedoch, dass wir
in unserer zukünftigen Arbeitslaufban mit guter Wahrscheinlichkeit auch irgend einmal
in solch eine Situation geraten könnten. Deshalb erschien uns dieser Weg trotzdem sinnvoll.

\begin{center}
  \begin{longtable}{r p{13cm}}
  \speakcc  Hallo \herrl, ich habe etwas mit dir zu besprechen. \\
            \commenting{\textit{würde ich so lassen}}\\
  \speakl   Guten Tag Herr Caesar, wie kann ich Ihnen helfen?\\
            \commenting{\textit{würde ich auch so lassen, da beide Zeilen mögliche
                        Gesprächsstarts sind}}\\

  \speakcc  Du hast gestern deine Arbeit abgeliefert, wie zufrieden bist du damit?\\
  \speakl   Ich hatte schwierigkeiten, aber bin der Meinung,
              ich habe es gut hinbekommen.\\

  \speakcc  Ist das dein Ernst?\\
            \commenting{\textit{je nach Körpersprache/Ton ist eine andere Antwort von nöten}}\\
  \speakl   ääähh\\
            \commenting{\textit{Die Äusserung bringt das Gespräch überhaupt nicht weiter! Besser wäre die unbewuste
                        Stimmung von CC in das Bewustsein von CC rufen. Emotionen im
                        Unterbewusten sind gefährlich (siehe Weissbach; Bsp. mit: möglichst
                        Natürlich vom Stuhl aufstehen und bequem stehen bleiben.)  Besser
                        z.B.:}}\\
            \commenting{Ich merke, dass Sie das Überrascht/Erzürnt, wie sehen Sie den meine Arbeit?}\\
            \commenting{\textit{Zweiter Satzteil ist nötig, um sein eigenes Bild richtig zu stellen.
                        Siehe Weissbach ?}}\\

  \speakcc  Du kannst ja nicht ernsthaft damit zufrieden sein.\\
            \commenting{\textit{mögliche neue Reaktion von CC:}}\\
            \commenting{Ich finde deine Arbeit schrecklich \emph{richtig in Fahrt kommend;
                        Schockiert; totales Unverstäntnis ins Gesicht geschrieben}}\\
  \speakl   Was meinen Sie damit? \\
            \commenting{\textit{Weiter versuchen die unbewusten Gefühle ins Bewustsein
                        zu rufen. $\rightarrow$ Weitere Informationen einholen, damit man sich
                        ein KORREKTES Bild des Problems machen kann. z.B.:}}\\
            \commenting{Ich bin verblüft wie schockiert Sie wirken.  Dann muss der Fehler
                        wohl ziemlich schlimm sein.  Ich möchte Sie gerne fragen, ob es
                        etwas mit der Funktion zu tun hat, die habe ich nämlich extra
                        noch getestet.}\\

  \speakcc  Als ich diese Arbeit sah, war ich schockiert.
              Die essentiellen Grundlagen waren überhaupt nicht eingehalten. \\
            \commenting{weg}\\
  \speakl   Der Schaltplan wurde doch von Ihnen korrigiert, oder?\\
            \commenting{weg}\\

  \speakcc  Natürlich, aber der war auch korrekt und das ist nicht das Problem hier!\\
            \commenting{Die Funktion sollte schon stimmen.  Ich habe den Schaltplan
                        ja kontrolliert.  Aber die Layoutumsetzung ist ja schrecklich.}\\
  \speakl   An was liegt es denn?!..\\
            \commenting{weg}\\

  \speakcc  Das Layout war ja unter aller Sau.
              Hast du denn noch nichts gelernt bei uns?\\
            \commenting{weg}\\
  \speakl   \emph{("geschocktes schweigen")}\\
            \commenting{Was ist in Ihren Augen nicht korrekt?}\\
  \speakcc  Sieh doch mal diese Stelle an, siehst du denn hier den Fehler nicht?\\
            \commenting{lassen}\\

  \speakl   Ehrlich gesagt, nein. Das wurde mir in der kurzen Zeit nicht erklärt.\\
            \commenting{\textit{Ehrlich gesagt, nein. Lassen, direkt sagen was man denkt
                        Weissbach?}}\\
            \commenting{Es war das erste Mal, dass ich solch eine Arbeit durchgeführt habe.
                        In dieser kurzen Zeit hatte ich noch keine Einführung.}\\ 

 \speakcc   Ja, aber wie du zum Beispiel die Winkel der Leiterbahnenecken gemacht hast.
              Da wird es ja den Elektronen schlecht!\\
            \commenting{Wieso hast du nicht zu Beginn gefragt?}\\
            \commenting{\textit{Dies könnte zu einer weiteren unglücklichen Diskusion
                        führen. Für die Entschärfung davon, würde nun ein Weissbachscher
                        Dreiteiler möglicherweise helfen. Erklährung einfügen.}}\\
 
  \speakl   \emph{(verwirrt)} Soll ich den Kurven machen?\\
            \commenting{Ich kann verstehen, dass das besser gewesen wäre. Ich möchte Sie
                        aber fragen, was Sie in meiner Position getan hätten? Denn ich
                        wollte diese Arbeit so selbständig wie möglich durchführen.}\\
  \speakcc  Sicher nicht!\\
            \commenting{\textit{Mögliche Erwiederung}}\\
            \commenting{Ja, dass war natürlich ein guter Gedanke. In Zukunft empfehle
                        ich dier, aber bei ganz neuen Arbeiten trotzdem zuerst zu fragen.
                        Das wirkt nämlich nicht unsälbständig.}\\

  \speakl   Dann sagen Sie mir doch, wie denn?\\
            \commenting{Werde ich machen!}\\
  \speakcc  In welchem Lehrjahr bist du denn eigentlich?\\
            \commenting{weg}\\
  \speakl   Im Ersten.\\
            \commenting{weg}\\
  \speakcc  Wieso hast du das nicht von Anfang an gesagt! \emph{(gehässig)}\\
            \commenting{weg}
  \end{longtable}
\end{center}