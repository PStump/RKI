\section*{Wieso Gespräche schlecht laufen}
\subsection*{Einleitung}
Unser Gesprächsszenario handelt von zwei Personen, Ludwig und Cristian Caesar. Ludwig absolviert eine Lehre als Elektroniker und wird als \hochkommas{Lehrling} oder in unserem Dialog schlicht mit \hochkommas{L}bezeichnet. Christan Caesar ist der Chef des direkten Vorgesetzten des Lehrlings. Er wird auch als \hochkommas{Chefchef}oder im Dialog als \hochkommas{CC}bezeichnet. Diese zwei Personen geraten in dieser Gesprächssituation aneinander.
\subsection*{Personenbeschreibung}
\begin{tabular}{p{150pt}p{280pt}}
Ludwig (Lehrling):	&
Ludwig ist 15 Jahre alt und von schüchterner Natur. Er kommt aus der Region Ostschweiz und absolviert eine Lehre als Elektroniker. Er befindet sich momentan im ersten Lehrjahr.\\ 
Chistian Caesar (Chefchef): & 
Christian Caesar ist wie schon vorhin erwähnt  der Chef des Lehrlinsausbildners und 48 Jahre alt. Seiner ambitionierten und ehrgeizigen Persönlichkeit verdankt er seine Stellung in diesem Unternehmen. Als Chef erwartet er von seinen 		Mitarbeitern vollen Einsatz und Präzision.
\end{tabular}
\subsection*{Gesprächserläuterung}
Zu Beginn des Gesprächs begegnen sich die beiden Parteien und der Chefchef erkundigt sich nach der abgegebenen Arbeit. Der Chefchef macht den Lehrling auf essentielle Fehler aufmerksam, dieser hingegen kann mit der Kritik nicht viel anfangen. In diesem Moment fühlt sich der Chefchef in seiner Person attackiert. Als er genauer auf die Fehler eingeht und im Lehrling die Ratlosigkeit steigt versucht dieser den Vorgesetzten auf seine Fehlende Erfahrung in diesem Gebiet hinzuweisen, wird allerdings überhört. Erst nach dem der Lehrling eine aus Sicht des Chefchefs eine völlig unqualifizierte Frage stellt erkundigt sich dieser nach dessen Wissensstand. Völlig entsetzt muss er feststellen dass er es mit einem unerfahrenen Lehrling aus dem ersten Lehrjahr zu tun hat. 
\subsection*{Verhalten der Beteiligten}
Der Lehrling Ludwig ist durch die Position seines gegenüber eingeschüchtert. Er gerät immer mehr in Erklärungsnot.\\
Christian Caesar, der Chefchef versucht zu Anfang Ludwig auf seine Fehler aufmerksam zu machen. Im Verlaufe des Gesprächs basieren die Reaktionen von Caesar jedoch immer mehr auf Emotionen.\\

