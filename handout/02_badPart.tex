\section*{Wieso Gespräche schlecht laufen}
\subsection*{Einleitung}
Unser Gesprächsszenario handelt von zwei Personen, Ludwig und Cristian Cesar. Ludwig absolviert eine Lehre als Elektroniker und wird in unseren Unterlagen schlicht mit L bezeichnet. Christan Cesar ist der Chef des direkten Vorgesetzten des Lehrlings und wird fortan als "Chefchef" bezeichnet. Diese zwei Personen Geraten in dieser Gesprächssituation aneinander.
\subsection*{Personenbeschreibung}
\begin{tabular}{p{150pt}p{280pt}}
Ludwig (Lehrling):	&
Ludwig ist 15 Jahre alt und von schüchterner Natur. Er kommt aus der Region Ostschweiz und absolviert eine Lehre als Elektroniker.\\ 
Chistian Cesar (Chefchef): & 
Christian Cesar ist wie schon vorhin erwähnt  der Chef des Lehrlinsausbildners und 48 Jahre alt. Seiner ambitionierten und ehrgeizigen Persönlichkeit verdankt er seine Stellung in diesem Unternehmen. Als Chef erwartet er von seinen 		Mitarbeitern vollen Einsatz und Präzision.
\end{tabular}

\subsection*{Gesprächserläuterung}
Zu Beginn des Gesprächs begegnen sich die beiden Parteien und der CC erkundigt sich nach der abgegebenen Arbeit. CC macht L auf essentielle Fehler aufmerksam, dieser hingegen kann mit der Kritik nicht viel anfangen. In diesem Moment fühlt sich CC in seiner Person attackiert. Als CC genauer auf die Fehler eingeht und in L die Ratlosigkeit steigt versucht dieser CC auf seine Fehlende Erfahrung in diesem Gebiet hinzuweisen, wird allerdings überhört. Erst nach dem, L eine aus CC's Sicht eine völlig unqualifizierte Frage stellt erkundigt sich dieser nach dessen Wissensstand. Völlig entsetzt muss er feststellen dass er es mit einem Lehrling aus dem ersten Lehrjahr zu tun hat. 
\subsection*{Verhalten der Beteiligten}
L: 	Ludwig ist durch die Position seines gegenüber eingeschüchtert. Er gerät immer mehr in Erklärungsnot.\\
CC:	CC versucht zu Anfang L auf seine Fehler aufmerksam zu machen. Im Verlaufe des Gesprächs basieren die Reaktionen von CC immer mehr auf Emotionen.\\

