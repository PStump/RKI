\subsection*{Spontanität}
Das natürliche Verhalten einer Person verliert schlagartig an Spontanität, sobald es in das Bewusstsein dringt. Nach Weissbach \textit{(Kapitel 18; S.\,375)} gibt es keine bewusste Spontanität. Ein Beispiel dazu ist, wenn man jemanden einen Witz erzählt und diesen mit den Worten \glqq dabei wirst du dich totlachen\grqq\text{ }einleitet. Man ruft dabei die Spontane Aktion des Lachen in das Bewusstsein und verhindert es dadurch. Es ist sehr schwierig ein spontanes Verhalten zu erzwingen.
\subsection*{aktives Zuhören}
Aktives Zuhören stillt das Primärbedürfnis: Ich will ernst genommen und verstanden werden. Das Wissen über die Spontanität wird dabei angewandt. Wenn man sich den momentanten Gefühlen bewusst macht, ist man diesen nicht auf so stark ausgeliefert. Die Bewusstheit setzt eine Person in die Lage, dass sie zu Ihren Gefühlen eine gewisse Distanz einnimmt. Es ist bei einem professionellen Gespräch daher wichtig, dass spontanes Verhalten unterdrückt wird.\\
Bei einer entgegennahme einer Beschwerde bei einer Person die sehr aufgebracht ist, ist es daher hilfreicher, wenn man anstatt Probiert die Person zu beruhigen mit: \glqq Nun beruhigen Sie sich doch erst mal\dots\grqq, die Person ernst nimmt und die Gefühle in das Bewusstsein ruft \glqq Sie sind empört, weil\dots\grqq\\
\\
Im konstruktiven Dialog wird dies angewendet, als Herr Caesar fragt, ob Ludwig den
Fehler an der gezeigten Stelle nicht sieht.  Da Ludwig keine Ahnung
hat, was der Fehler sein könnte und Ludwig weiss, dass Herr Caesar
negativ reagieren könnte.  Versucht Ludwig das negative Verhalten zu
umgehen, indem er seine Antwort beginnt mit:  \glqq Auch wenn ich Sie
vielleicht noch mehr aus der Fassung bringe. \dots\grqq

