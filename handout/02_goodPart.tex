\section*{Wie man mit schlechten Emotionen umgeht}
\subsection*{Spontanität}
Das natürliche Verhalten einer Person verliert schlagartig an
Spontanität, sobald es in das Bewusstsein dringt. Nach Weissbach
\textit{(Kapitel 18; S.\,375)} gibt es keine bewusste Spontanität. Ein
Beispiel dazu ist, wenn man jemanden einen Witz erzählt und diesen mit
den Worten \glqq dabei wirst du dich totlachen\grqq\text{
}einleitet. Man ruft dabei die spontane Aktion des Lachen in das
Bewusstsein und verhindert es
dadurch. %Es ist sehr schwierig ein spontanes Verhalten zu erzwingen.

\subsection*{aktives Zuhören}
Aktives Zuhören stillt das Primärbedürfnis: Ich will ernst genommen
und verstanden werden. Das Wissen über die Spontanität wird dabei
angewandt. Die Bewusstheit setzt eine Person in die Lage, dass Sie zu
Ihren Gefühlen eine gewisse Distanz einnimmt. Es ist bei einem
professionellen Gespräch daher wichtig, dass spontanes Verhalten
unterdrückt wird.

Bei einer entgegennahme einer Beschwerde bei einer Person die sehr
aufgebracht ist, ist es kontraproduktiv, wenn man versucht die Person
zu beruhigen, zum Beispiel mit: \glqq Nun beruhigen Sie sich doch erst
mal\dots\grqq. Wesentlich hilfreicher wäre es die Person ernst zu
nehmen und die Gefühle in ihr Bewusstsein ruft: \glqq Sie sind empört,
weil\dots\grqq.

Im konstruktiven Dialog wird dies angewendet, als Herr Caesar fragt,
ob Ludwig den Fehler an der gezeigten Stelle nicht sieht.  Da Ludwig
keine Ahnung hat, was der Fehler sein könnte und Ludwig weiss, dass
Herr Caesar negativ reagieren könnte.  Versucht Ludwig das negative
Verhalten zu umgehen, indem er seine Antwort beginnt mit: \glqq Auch
wenn ich Sie vielleicht noch mehr aus der Fassung bringe\dots\grqq

\subsection*{Sein eigenes Bild}
Bei Diskussionen, ist es immer wichtig, sein eigenes Bild dem
Gesprächspartner richtig zu vermitteln.  Umgekehrt ist es auch wichtig
das Bild des Gesprächspartner richtig zu erhalten.  Dazu ist es
wichtig, dass man während der Diskussion immer wieder nachfragt bis
man sicher ist, dass richtige Bild sich vorzustellen. \textit{(Kapitel
1; S.\,3)}

Zum Beispiel hat Ludwig das Bild, dass die Funktion seiner
Arbeit das Problem ist. Durch nachfragen merkt er jedoch, dass die
Layoutumsetzung das Problem der Diskussion ist.

\subsection*{Überzeugen}
Gut Überzeugen kann man mit einem Dreischritt.  Bei welchem man mit
einer Bestätigung den Gesprächspartner dort abholt, wo er sich gerade
befindet.  Gefolgt von einer Frage, welche in die richtige Richtung
zielt.  Abgeschlossen mit der Begründung, welche dem Gesprächspartner
die nötige Sicherheit gibt. \emph{(Kapitel 16; S.\,344)}

Im Dialog Umgesetzt, als der Chefchef fragte, wieso Ludwig nicht zu
Beginn gefragt hatte, wie die Arbeit korrekt zu erledigen sei:
\begin{description}
  \item[Bestätigung:] Ich kann nachträglich verstehen, dass das besser
    gewesen wäre.
  \item[Frage:] Ich möchte Sie aber fragen, was Sie in meiner Position
    getan hätten?
  \item[Begründung:] Denn ich wollte dies Arbeit so selbständig wie
    möglich durchführen.
\end{description}




%%% Local Variables: 
%%% mode: latex
%%% TeX-master: "01_handoutMain"
%%% End: 
