\section{Konstruktiver Dialog}
\herrcc: Chef-Chef; \herrl: Lehrling\\

\begin{center}
  \begin{longtable}{r p{13cm}}
  \speakcc  Hallo \herrl, ich habe etwas mit dir zu besprechen. \\
  \speakl   Guten Tag Herr Caesar, wie kann ich Ihnen helfen?\\

  \speakcc  Du hast gestern deine Arbeit abgeliefert, wie zufrieden bist du damit?\\
  \speakl   Ich hatte schwierigkeiten, aber bin der Meinung,
              ich habe es gut hinbekommen.\\

  \speakcc  Ist das dein Ernst?\\
  \speakl   Ich merke, dass Sie das Überrascht, wie sehen Sie denn meine Arbeit?\\

  \speakcc Ich finde deine Arbeit schrecklich. \emph{(totales
    Unverständnis ins Gesicht geschrieben)}\\
  \speakl Ich bin verblüfft wie schockiert Sie wirken.  Ich möchte Sie
  gerne fragen, ob es etwas mit der Funktion zu tun hat, die habe ich
  nämlich extra noch getestet.\\

  \speakcc  Die Funktion sollte schon stimmen.  Ich habe den Schaltplan ja kontrolliert.
              Aber die Layoutumsetzung ist ja schrecklich.\\
  \speakl   Was ist in Ihren Augen nicht korrekt?\\

  \speakcc  Sieh doch mal diese Stelle an, siehst du denn hier den Fehler nicht?\\
  \speakl Auch wenn ich Sie vielleicht noch weiter aus der Fassung
  bringe.  Muss ich leider sagen, dass ich den Fehler nicht sehe.  Es
  war das erste Mal, dass ich solch Arbeit durchgeführt habe.  In
  dieser kurzen Zeit hatte ich noch keine Einführung.\\

  \speakcc  Wieso hast du nicht zu Beginn gefragt?\\
  \speakl   Ich kann nachträglich verstehen, dass das besser gewesen wäre.  Ich möchte Sie aber fragen,
              was Sie in meiner Position getan hätten?  Denn ich wollte diese Arbeit so
              selbständig wie möglich durchführen.\\

  \speakcc  Ja, dass war natürlich ein guter Gedanke.  In Zukunft empfehle ich dir aber,
              bei ganz neuen Arbeiten trotzdem zuerst zu fragen.  Das wirkt
              nämlich nicht unselbständig.\\

  \speakl   Werde ich machen!\\
  \end{longtable}
\end{center}