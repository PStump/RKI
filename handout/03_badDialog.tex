\section{Ausgangsdialog}
\herrcc: ChefChef; \herrl: Lehrling\\

\begin{center}
  \begin{tabular}{r p{13cm}}
  \speakcc  Hallo \herrl, ich habe etwas mit dir zu besprechen. \\
  \speakl   Guten Tag Herr Caesar, wie kann ich Ihnen helfen?\\

  \speakcc  Du hast gestern deine Arbeit abgeliefert, wie zufrieden bist du damit?\\
  \speakl   Ich hatte schwierigkeiten, aber bin der Meinung,
              ich habe es gut hinbekommen.\\

  \speakcc  Ist das dein Ernst?\\
  \speakl   ääähh\\

  \speakcc  Du kannst ja nicht ernsthaft damit zufrieden sein.\\
  \speakl   Was meinen Sie damit? \\

  \speakcc  Als ich diese Arbeit sah, war ich schockiert.
              Die essentiellen Grundlagen waren überhaupt nicht eingehalten. \\
  \speakl   Der Schaltplan wurde doch von Ihnen korrigiert, oder?\\

  \speakcc  Natürlich, aber der war auch korrekt und das ist nicht das Problem hier!\\
  \speakl   An was liegt es denn?!..\\

  \speakcc  Das Layout war ja unter aller Sau.
              Hast du denn noch nichts gelernt bei uns?\\
  \speakl   \emph{("geschocktes schweigen")}\\
  \speakcc  Sieh doch mal diese Stelle an, siehst du denn hier den Fehler nicht?\\

  \speakl   Ehrlich gesagt, nein. Das wurde mir in der kurzen Zeit nicht erklärt.\\
  \speakcc  Ja, aber wie du zum Beispiel die Winkel der Leiterbahnenecken gemacht hast.
              Da wird es ja den Elektronen schlecht!\\

  \speakl   \emph{(verwirrt)} Soll ich den Kurven machen?\\
  \speakcc  Sicher nicht!\\

  \speakl   Dann sagen Sie mir doch, wie denn?\\
  \speakcc  In welchem Lehrjahr bist du denn eigentlich?\\

  \speakl   Im Ersten.\\
  \speakcc  Wieso hast du das nicht von Anfang an gesagt! \emph{(gehässig)}
  \end{tabular}
\end{center}