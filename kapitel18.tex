\section{18. Kapitel \page{375} - Umgang mit negativen Emotionen}
Unser natürliches Verhalten verliert schlagartig seine Spontanität, sobald es in unser Bewusstsein dringt oder anders formuliert: \textbf{ES gibt keine bewusste Spontanität} \page{375}.\\
\\
Benennen Sie Handlungsweisen, die sie \emph{verhindern} wollen. Heben Sie spontanes Verhalten ins Bewustsein \page{378}.\\
\bsp{\glqq Bei dem Witz werdet ihr euch totlachen.\grqq - Niemand wird lachen!}

Mit aktivem Zuhören stillen Sie das Primärbedürfnis: \glqq Ich will ernst genommen und verstanden werden.\grqq\\
Können wir uns unsere momentann Gefühle bewusst machen, sind wir ihnen nicht mehr auf Gedeih und Verderb ausgeliefert.\\
Die \emph{Bewusstheit} versetzt uns in die Lage zu unseren Gefühlen eine gewisse Distanz einzunehmen\page{379}\\
\\
Wer seine Erwartung in einer bestimmten Sache nicht erfüllt sieht, reagiert spontan mit einem negativen Gefühl, ist unzufrieden und verstimmt.
\bsp{wenn sich eine gewisse \emph{Hilflosigkeit} einstellt, sobald wir feststellen, dass wir das Problem nicht aus eigener Kraft zufriedenstellend lösen können.}
\\
Wenn Sie mit Beschwerden konfrontiert werden, vergegenwärtigen Sie sich, in welcher negativen Verfassung sich Ihr Gesprächspartner befinden könnte.\page{382}\\
\\
Das Denken erweist sich im emotional aufgewühlten Zustand als beeinträchtigt. \\
Der Wunsch, ernst genommen zu werden, ist auf beiden Seiten gleich gross.\page{383}\\
\\
Nach Abschluss eines Reklamationsgesprächs ergeben sich vier Zustände. Man fühlt sich , hat das Ziel erreicht oder das Ziel nicht erreicht. Oder man fühlt sich schlecht, hat das Ziel erreicht oder nicht erreicht. \\
Fehlende Wertschätzung lässt sich nicht durch sachliche Zugeständnisse ersetzen. Daraus folgt: Wer erhält, was er möchte, muss sich deswegen noch lange nicht wohl fühlen.\page{384}\\
\\
Reklamationen fussen in der Regel nicht auf objektiven Einschätzungen der Sachlage, sondern auf einem \emph{Gefühl, schlecht behandelt worden zu sein.} Und wer sich ungerecht behandelt fühlt, neigt zu ungerechtem Verhalten und dazu, enstandene Schwierigkeiten zu \emph{dramatisieren}. Daraus folgt: Je stärker die emotionale Betroffenheit, umso weniger erfolgreich ist eine rein sachliche, auf Fakten bezogene Lösung. \\
Durch gezielte Wertschätzung können Sie ein sachliches Zugeständnis gering halten oder gar ersetzen. Daraus folgt: Wer nicht erhält, was er möchte, muss sich deswegen noch lange nicht unwohl fühlen.\page{388}\\
Wenn Sie Beschwerden bearbeiten, werden Sie oftmals als \glqq Blitzableiter \grqq benutzt, und in dieser Funktion gilt es, nicht nur die konkrete Beanstandung zu bearbeiten, sondern sich auch dem verletzten Gefühlen des Gesprächspartners zuzuwenden.\page{391}.\\
\\
Es ist möglich die Gesprächsatmosphäre förderlich zu beinflussen. Es schafft ganz grundsätzlich eine positive Stimmung, wenn Sie dem Kunden Recht geben, 
\begin{itemize}
\item weil er Wiedrigkeiten hatte.
\item weil er sofort seine Beanstandung vorträgt.
\item weil er von Ihnen eine Lösung erwartet.
\item weil er für sein Problem eine schnelle Bearbeitung wünscht. 
\item weil er auf eine Dauerlösung dringt.
\item weil er zukünftigen Schaden abwenden will.
\end{itemize}
voreiliges Reagieren und beipflichten kann dazu führen, dass der Kunde den Eindruck bekommt, man möchte ihn so schnell wie möglich abwimmeln. Erst wenn der Reklamierer spürt, dass Sie ihn wirklich ernst nehmen, kann ein positives Gesprächsklima entstehen, bei dem das zerstörte Vertrauen wiederaufgebaut wird.\\
Wenn Sie das, was Sie im Begriffe stehen, beim anderen auszulösen bereits sprachlich vorwegnehmen, bringen Sie Ihrem Gesprächspartner Verständnis entgegen\page{397}.
\bsp{\glqq Vielleicht stosse ich Sie jetzt vor den Kopf, wenn ich mich jetzt wieder meiner Arbeit zuwende. Ich möchte gern hiermit fertig werden. \grqq}
Offenheit und Bereitschaft zur Einfühlung macht diese Gesprächsstrategie unschlagbar. Ehrlichkeit hat etwas entwaffnendes.\\
\bsp{\glqq Vielleicht klingt das in Ihrem Ohr unglaublich.\grqq}\\
Diese Technik kann man als \glqq Stier-bei-den-Hörnern-packen \grqq bezeichnen. Der andere wird wirklich ernst genommen und hat es nicht länger nötig, pseudosachlich zu wiedersprechen.\page{401} Damit raubt man einem unwillkürlich einstellenden Gefühl seine Spontanität, wenn Sie ausdrücklich zubilligen, dass es gleich entstehen kann. \\
Der Aber-Satz verneint den vorgegangenen Satz.\page{404}
\bsp{\glqq Ja, da haben Sie vollkommen Recht, \emph{aber} in diesem Fall verhält es sich doch ganz anders... \grqq}
Wenn man sich vergegenwärtigt, dass Wertschätzung des anderen eine Voraussetzung erfolgreicher Gesprächsführung ist, werden wird das klassische, den anderen gering schätzende Muster: \\
ja, Du willst x, aber ich will y.\\
in ein wertschätzendes Muster eintauschen:\\
Ich will x, aber Du willst y.\\
Durch Wertschätzung und Ernstnehmen seiner Gesprächspartner wird \emph{keinenfalls} die eigene Position aufgegeben.\\
Mit dem Austausch des Wortes \glqq aber \grqq durch das Wort \glqq und\grqq können Sie Konflikte reduzieren. Sie zeigen damit, dass zwei gegensätzliche Vorstellungen gleichzeitig nebeneinander existieren können.\page{406}\\
\\
Mithilfe von drei Punkten kann man bei einer professionellen Gesprächsführung zur emotionalen Sicherheit der Gesprächspartner beitragen. \\
Klären Sie zu Beginn eines Gesprächs Ihr Gegenüber über den \emph{Gesprächsumfang}, Ihre \emph{Ziele} und Ihre konkreten \emph{Erwartungen} auf.\page{409}.\\
Wer konstruktiv kritisiert, überlegt sich das wünschenswerte Zielverhalten\page{413}.\\

